% Melody is a fundamental aspect of music perception. 
While even those without musical training can intuitively recognize the melody of a song by ear, 
it remains an open challenge in MIR to reliably detect the notes of the melody present in an arbitrary music recording. 
A key challenge in \emph{melody transcription} is building methods which can handle broad audio containing any number of instrument ensembles and musical genres. 
To confront this challenge, we leverage recent advancements in generative modeling of broad music audio, thereby improving performance on melody transcription by 
% (RWC All) .744 vs .631 = 17.9%
% (Hookthr) .615 vs .514 = 19.6%
% (RWC Vox) .786 vs .621 = 26.6%
up to $27$\% 
relative to conventional approaches. 
Another obstacle in melody transcription is a lack of training data---we collect, align, and release a new dataset consisting of $50$ hours of crowdsourced melody annotations for popular music. 
% (RWC Vox) 0.786 vs 0.462 = 70%
% (RWC All) 0.744 vs 0.420 = 77%
The combination of generative pre-training and a new dataset for this task results in up to $77\%$ stronger performance on melody transcription relative to the strongest available baseline. 
By pairing our new melody transcription approach with solutions for beat detection, key estimation, and chord recognition, 
we build a system capable of transcribing human-readable lead sheets directly from music audio.\footnote{Sound examples: \url{https://dblblnd.github.io/ismir22} \\
Demo / dataset explorer: \url{https://colab.research.google.com/drive/1yzD3wRCjXkuSfDRUtt_daaGFitj5L88l}\label{sound_examples}}