Despite the central role that melody plays in music perception, 
it remains an open challenge in MIR to reliably detect the notes of the melody present in an arbitrary music recording. 
A key challenge in \emph{melody transcription} is building methods which can handle broad audio containing any number of instrument ensembles and musical styles---existing strategies work well for \emph{some} melody instruments or styles but not all. 
To confront this challenge, we leverage representations from \jukebox{}~\cite{dhariwal2020jukebox}, 
a generative model of broad music audio, 
thereby improving performance on melody transcription by 
% (RWC All) .744 vs .631 = 17.9%
% (Hookthr) .615 vs .514 = 19.6%
% (RWC Vox) .786 vs .621 = 26.6%
$20$\% 
relative to conventional spectrogram features. 
Another obstacle in melody transcription is a lack of training data---we derive a new dataset containing $50$ hours of melody transcriptions from crowdsourced annotations of broad music. 
% (RWC Vox) 0.786 vs 0.462 = 70%
% (RWC All) 0.744 vs 0.420 = 77%
The combination of generative pre-training and a new dataset for this task results in 
$77\%$ stronger performance on melody transcription relative to the strongest available baseline.\footnote{Examples: \url{https://chrisdonahue.com/sheetsage} \\
Code: \url{https://github.com/chrisdonahue/sheetsage}
\label{sound_examples}} 
By pairing our new melody transcription approach with solutions for beat detection, key estimation, and chord recognition, 
we build Sheet Sage, a system capable of transcribing human-readable lead sheets directly from music audio.