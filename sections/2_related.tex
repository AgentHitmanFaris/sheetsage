\section{Related work}\label{sec:related}

% Melody transcription is often cast as a melody \emph{extraction} task, which requires us to annotate an audio recording with a time-varying, continuous fundamental frequency (F0) estimate \cite{goto2004real}. 
Melody transcription is closely related to but distinct from the task of \emph{melody extraction}, originally referred to as predominant fundamental frequency (\fnot) estimation~\cite{goto1999real,goto2004real}. 
The goal of melody extraction is to estimate the time-varying, continuous \fnot{} trajectory of the melody in an audio mixture. 
% CHRIS: I think we shouldn't talk about lead sheets as our goal here. We're pitching lead sheets purely as a bonus (since I failed to evaluate them in time)
% In contrast, our work is concerned with transcription to human-readable \emph{lead sheets}: this task requires us to indicate the positions of notes within a metrical structure, belonging to discrete pitch classes.
In contrast, the goal of melody transcription is to output the \emph{notes} of the melody, where a note is defined by an onset time, a discrete pitch, and an offset time. 
% Melody extraction has been popularized by annual results of the MIREX Audio Melody Extraction tasks \cite{downie2014ten}, 
% garnering significant interest from the MIR community over the last two decades \cite{salamon2014melody,rao2022melody}. 
Melody extraction has received significant interest from the MIR community over the last two decades (see~\cite{salamon2014melody,rao2022melody} for comprehensive reviews), 
and is the subject of an annual MIREX competition~\cite{downie2014ten}. 
However, 
while melody extraction is useful for several downstream tasks (e.g.,~query-by-humming) and more inclusive of music which does not use equal temperament, 
its outputs cannot be readily converted into familiar formats like MIDI or scores. 
%, which can be readily converted to symbolic formats. 
Melody extraction may be a component of a melody transcription pipeline in combination with a strategy to segment F0 
%trajectories 
into notes~\cite{salamon2015midi,nishikimi2016musical,nishikimi2017scale}---we directly compare to such a pipeline.

%Lead-sheet melody transcription has received considerably less attention. 
Compared to melody extraction, melody transcription has received considerably less attention. 
Earlier efforts use sophisticated DSP-based pipelines~\cite{paiva2004auditory,paiva2005detection,ryynanen2008automatic,weil2009automatic}---unfortunately none of these methods provide code, though~\cite{ryynanen2008automatic} provides example transcriptions which we use to facilitate direct comparison. 
A more recent effort uses ground truth chord labels as extra information to improve melody transcription~\cite{laaksonen2014automatic}---in contrast, our method does not require extra information. 
% Poliner et. al. observed in 2007 that ``an attempt was made to evaluate the lead voice transcription at the lowest level of abstraction [melody extraction], and as such, the concept of segmenting the fundamental frequency predictions into notes has been largely omitted from consideration'' \cite{poliner2007melody}. 
% This omission largely remains true today. 
Another line of work seeks to transcribe solo vocal performances into notes~\cite{mauch2015computer,nishikimi2020bayesian,nishikimi2021audio}. 
As singing voice often carries the melody in popular music, we directly compare to a baseline which firsts isolates the vocals (using~\cite{hennequin2020spleeter}) and then transcribes them.

Polyphonic music transcription is another related task which involves transcribing \emph{all} of the notes present in a recording (not just the melody).
This task has its own MIREX contest (Multiple Fundamental \fnot{} Estimation) alongside a growing collection of supervised training data resources \cite{benetos2013automatic,thickstun2017learning,hawthorne2019enabling,manilow2019cutting}. 
% Like melody transcription, polyphonic transcription is typically formalized as a low-level \emph{frame-based} transcription task, which requires us to annotate an audio recording with a time-aligned piano roll. 
% Like melody extraction, this task does not result in a human-readable lead sheet or score. 
% Nevertheless,
The similarity of the polyphonic and melody transcription problems motivates us to
%, together with recent progress towards polyphonic transcription, 
% consider whether representations learned by a polyphonic music transcription system---specifically, MT3 \cite{gardner2021mt3}---are useful for melody transcription.
% CHRIS: Rewrote this because the above doesn't make it clear that we actually try this
experiment with using representations learned by a polyphonic music transcription system---specifically, MT3 \cite{gardner2021mt3}---for melody transcription.